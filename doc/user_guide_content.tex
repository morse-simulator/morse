\documentclass[twoside,a4paper,10pt]{report}
%%\usepackage[french]{babel} %% Use your own babel language
\usepackage[top=3cm,bottom=3cm,left=2.5cm,right=2.5cm]{geometry}
\usepackage{ucs}
\usepackage[utf8x]{inputenc}
\usepackage{pslatex}
\usepackage{hyperref}
\usepackage{graphicx}
\usepackage{tabularx}
\usepackage{supertabular}
\usepackage{pdflscape} %% Used for very big table
\usepackage{moreverb}
\usepackage{xcolor}
\usepackage{listings}
\usepackage{lastpage}
\usepackage{fancyhdr}
\usepackage{ulem}
\usepackage{textcomp}
\usepackage{wasysym}
\usepackage{sectsty}
\usepackage{wrapfig} %%Usefull for image 
%\usepackage{fguill} %%Use this package for guillemot[left|right] useless with babel set to french
\usepackage{eso-pic} %% Background
\pagestyle{fancy}%
\renewcommand{\headrulewidth}{0.1pt}
\renewcommand{\footrulewidth}{0pt}
\renewcommand{\chaptermark}[1]{%
	\markboth{\sffamily \chaptername\ \thechapter.\ #1}{}}

\renewcommand{\sectionmark}[1]{%
	\markright{{\sffamily #1}}{}}

\fancypagestyle{plain}{%
\fancyhf{}
\fancyfoot[R]{\thepage/\pageref{LastPage}}
\renewcommand{\headrulewidth}{0pt}
\renewcommand{\footrulewidth}{0pt}}
\fancyhf{}
\fancyhead[L]{\rightmark}
\fancyhead[R]{\leftmark}
\fancyfoot[R]{\thepage/\pageref{LastPage}}
\renewcommand{\headrulewidth}{0.1pt}
\renewcommand{\footrulewidth}{0pt}

\definecolor{Light}{gray}{.80}
\definecolor{Dark}{gray}{.20}

\newcommand{\key}[1]{\fcolorbox{Dark}{Light}{\textbf{#1}}}

\newcommand{\settexitref}[2]{(\ref{#1}p\pageref{#1})}
\newcommand{\dokutitlelevelone}[1]{\chapter{#1}}
\newcommand{\dokutitleleveltwo}[1]{\section{#1}}
\newcommand{\dokutitleleveltree}[1]{\subsection{#1}}
\newcommand{\dokutitlelevelfour}[1]{\subsubsection{#1}}
\newcommand{\dokutitlelevelfive}[1]{\paragraph{#1}}
\newcommand{\dokufootnote}[1]{\footnote{#1}}
\newcommand{\dokufootmark}[1]{\footnotemark[#1]}
\newcommand{\dokubold}[1]{\textbf{#1}}
\newcommand{\dokuitalic}[1]{\textsl{#1}}
\newcommand{\dokumonospace}[1]{\texttt{#1}}
\newcommand{\dokuunderline}[1]{\underline{#1}}
\newcommand{\dokuoverline}[1]{\sout{#1}}
\newcommand{\dokusupscript}[1]{\textsuperscript{#1}}
\newcommand{\dokusubscript}[1]{$_{#1}$}
\newcommand{\dokuhline}{\line(1,0){400}}
\newcommand{\dokulabel}[1]{\label{#1}}
\newcommand{\dokuitem}{\item}
\newcommand{\dokuquoting}{\textbar}
\newcommand{\dokutabularwidth}{\textwidth}
\newcommand{\dokusupertabularheadbreak}{\small\sl continued from previous page}
\newcommand{\dokusupertabulartailbreak}{\small\sl continued on next page}
\newcommand{\dokuheadingstyle}[1]{\textbf{#1}}
\definecolor{dokuheadingcolor}{rgb}{0,0,0.60}
\newcommand{\dokubackground}[1]{%
\AddToShipoutPicture{%
  \AtTextCenter{%
    \makebox(0,0)[c]{\resizebox{\textwidth}{!}{%
      \rotatebox{25}{\textsf{\textbf{\textcolor[gray]{0.90}{#1}}}}}}%
  }%
 }%
}

\graphicspath{{media/}}

\hypersetup{
pdftitle = {The Modular OpenRobots Simulation Engine - User guide},
pdfauthor = {LAAS-CNRS - ONERA},
pdfkeywords = {simulation, blender, robotics},
pdfcreator = {DokuTeXit},
pdfproducer = {dokuwiki + TeXit + pdflatex}
}
\title{The Modular OpenRobots Simulation Engine - User guide}
\author{LAAS-CNRS - ONERA}
\date{@DATE@}
\dokubackground{@BGTEXT@}
\begin{document}
\sffamily
\allsectionsfont{\sffamily}
\thispagestyle{empty}
\maketitle
\thispagestyle{empty}
\cleardoublepage
\tableofcontents
\newpage
\thispagestyle{plain}
\cleardoublepage
\newpage





\dokutitlelevelone{Introducing the Modular OpenRobots Simulation Engine}
\label{55d4061a5aa52cfc82c5474279f29bd9}%% introducing_the_modular_openrobots_simulation_engine
\label{a80da1282f2c775bbc5f2c92c836968b}%%Start: summary => /home/slemaign/openrobots/data/pages/summary.txt

img /home/slemaign/openrobots/data/media/morse/openrobots-simulator.png

Welcome to the official documentation for the MORSE project.

For an global description of the project, consult the 
\href{http://homepages.laas.fr/gechever/Documents/morse-21062010.pdf}{ article}
submitted to the \href{http://www.simpar.org/}{ Simpar 2010} conference.

This first section will help you to find your way into the MORSE documentation.


\dokutitleleveltwo{What is MORSE?}
\label{cb90401e2d53fdeab390406232e6c72f}%% what_is_morse

img /home/slemaign/openrobots/data/media/morse/simu{\textunderscore}render{\textunderscore}indoors.jpg



\begin{itemize}
\dokuitem  A versatile simulator for \dokubold{generic mobile robots simulation} (single or multi robots),
\dokuitem  Enabling \dokubold{realistic} and \dokubold{dynamic} environments (with other interacting agents -humans- or objects), 
\dokuitem  Don't reinvent the wheel: critical components reused from other opensource projects (\dokubold{Blender} for 3D rendering + physical simulation + UI, dedicated robotic middlewares for communications + robot hardware support),
\dokuitem  \dokubold{Seamless workflow}: since the simulator rely on Blender for both modelling and the realtime 3D engine, creating and modifying a simulated scene is straigthforward.
\dokuitem  Entierely scriptable in \dokubold{Python},
\dokuitem  Adaptable to various \dokubold{level of simulation realism} (for instance, we may want to simulate exteroceptive sensors like cameras in certain cases and access directly to a higher level representation of the world -like labelled artifacts- in other cases),
\dokuitem  Currently compatible with \dokubold{YARP} and LAAS OpenRobots robotics frameworks,
\dokuitem  Fully open source, BSD-compatible.
\end{itemize}

img /home/slemaign/openrobots/data/media/morse/outdoor{\textunderscore}example.jpg


\dokutitleleveltwo{Getting started}
\label{46b862bd16703034ef594b2996ec8166}%% getting_started

\begin{enumerate}\dokuitem  \hyperref[ea09bb364ef1bffd889e76b7a59035fc]{ Install MORSE}
\dokuitem  \hyperref[60efe788544a384827c39a9803dab85b]{ Check the MORSE command reference}
\dokuitem  \hyperref[0575c8d592fb7b088226750aceec2b4e]{ Jump to the tutorial}
\dokuitem  \hyperref[feb94730bf2c8bc6803a472bb56691ae]{ Prepare a robot with specific equipment}
\end{enumerate}

\dokutitleleveltwo{The MORSE Workflow}
\label{2963f1f90771a121484ee5fd4a4251c0}%% the_morse_workflow

Discover the MORSE workflow: how to build a complete simulation scenario, from 
the creation of a custom robot with predefined sensors and actuators to the 
complete scene, including other robots or humans.

\hyperref[514bac84019bd5e09c0e2b525b09f429]{ Go to: the typical MORSE workflow}


\dokutitleleveltwo{Components library \& Supported middlewares}
\label{0c88d1b20b693084703691e13ff5151f}%% components_library_supported_middlewares

\begin{itemize}
\dokuitem  MORSE offers a set of predefined sensors and controllers that cover basic simulation needs in robotics. To know how to add new components, please refer to the developer documentation.
\end{itemize}

The following page lists all the currently existing components and their properties:

\hyperref[004fdec0cc1a00c19c57e892b7eb1400]{ Go to: the MORSE component library}



\begin{itemize}
\dokuitem  The output (or input) of the simulator can be altered (for instance to add noise) by so called modifiers.
\end{itemize}

\hyperref[25bc6523e9298f4691b3c8200a395d92]{ Go to: Data modifiers}



\begin{itemize}
\dokuitem  MORSE relies on \dokuitalic{middlewares} to integrate in your robotic architecture.
\end{itemize}

We currently support only \dokubold{\href{http://eris.liralab.it/yarp/}{ YARP}}, 
\dokubold{\href{https://softs.laas.fr/openrobots/wiki/pocolibs}{pocolibs}} and a simple 
text-based socket protocol. More middlewares are expected to be added in the 
next versions.

\hyperref[9a05db9c4b60b0527010fd997682f523]{ Go to: Middleware support}


\dokutitleleveltwo{Advanced tutorials}
\label{1db3103f04a8f50e1168ef3c23748f71}%% advanced_tutorials

\hyperref[1db3103f04a8f50e1168ef3c23748f71]{ List of all tutorials}


\dokutitleleveltree{Setting up a YARP-based simulation}
\label{a3bba0b321b28de69351875f85d854db}%% setting_up_a_yarp-based_simulation

This tutorial shows a simple scenario with Yarp: Simple dummy autonomous navigation towards a user-given target (x,y). 
The robots becomes red when it intersects obstacles or bounces on them. Use the same example as the quickstart ? 
(sole difference: the goal is given through yarp, sensor data are exported with yarp  maybe add a camera, since it is trivial to displa an image with yarp).

\hyperref[1dd029a60f7f3dd1deaf993ce4538edf]{ Go to: YARP-based simulation tutorial}

{\dokuquoting}{\dokuquoting} \hyperref[a80da1282f2c775bbc5f2c92c836968b]{ User documentation}


\dokutitlelevelone{MORSE installation}
\label{1d96fd68defedd8a755f2a95c80e618f}%% morse_installation
\label{ea09bb364ef1bffd889e76b7a59035fc}%%Start: installation => /home/slemaign/openrobots/data/pages/installation.txt

\dokutitleleveltwo{Requirements - What you need to install before}
\label{27060cbab4a02c4805c03a15b2aad7d7}%% requirements_-_what_you_need_to_install_before

\dokutitleleveltree{Hardware}
\label{3ca14c518d1bf901acc339e7c9cd6d7f}%% hardware

To display textures correctly in the simulator, as well as to generate images using the simulated cameras, you will need to have a graphics card that supports GLSL shading. The Blender website lists these graphic cars as compatible with GLSL:


\begin{itemize}
\dokuitem  ATI Radeon 9{\texttimes}00, Xx00, X1x00, HD2x00 and HD3x00 series and newer.
\dokuitem  NVidia Geforce FX, 6{\texttimes}00, 7{\texttimes}00, 8{\texttimes}00, 9{\texttimes}00 and GTX 2{\texttimes}0 and newer.
\end{itemize}

\dokutitleleveltree{Required software}
\label{accfa4c836a5caff827d9adbf6bea7dc}%% required_software

\begin{itemize}
\dokuitem  Python (2.6 or +)
\dokuitem  Blender 2.49 build with Python 2.6 \dokufootnote{For the moment the simulator works only with this version. Blender 2.5 is being worked on.}
\dokuitem  git to get the code of the simulator:
\end{itemize}

\small
\begin{verbatimtab}
$ git clone http://trac.laas.fr/git/robots/morse.git
\end{verbatimtab}
\normalsize

If you plan to use the simulator with raw sockets of text files as "middleware",
you don't need anything else. Otherwise, you need to install the software for other middlewares.


\dokutitleleveltree{YARP}
\label{ec46d0b85077d7a7fe8da2e2b4c70462}%% yarp

For the YARP bindings


\begin{itemize}
\dokuitem  YARP version (2.2.5 or +)
\dokuitem  YARP python binding
\dokuitem  ACE ( 5.6.3 or +, required for YARP)
\end{itemize}

Instructions to create YARP-Python bindings are here: \href{http://eris.liralab.it/wiki/YARP_and_Python}{http://eris.liralab.it/wiki/YARP{{\textbackslash}textunderscore}and{{\textbackslash}textunderscore}Python}

Note that the easiest way to install YARP is probably to use \dokumonospace{robotpkg} (see \href{http://homepages.laas.fr/mallet/robotpkg}{ robotpkg homepage} for more informations). Follow the instructions on installing \dokumonospace{robotpkg}. Then add the environment variable \dokumonospace{ROBOTPKG{\textunderscore}BASE} to your shell.
Then to install \dokumonospace{yarp}:


\small
\begin{verbatimtab}
$ cd $ROBOTPKG_BASE/robotpkg/architecture/yarp
$ make update
\end{verbatimtab}
\normalsize
Afterwards, install the yarp python bindings bindings:


\small
\begin{verbatimtab}
$ cd $ROBOTPKG_BASE/robotpkg/devel/libpyyarp
$ make update
\end{verbatimtab}
\normalsize

Compiling the YARP Python binding will create two files: \dokumonospace{yarp.py} and \dokumonospace{{\textunderscore}yarp.so}, and install them in \dokumonospace{{\textdollar}ROBOTPKG{\textunderscore}BASE/lib/python2.6/site-packages/}
You'll need to set the environnement variable \dokumonospace{PYTHONPATH} to \dokumonospace{{\textdollar}ROBOTPKG{\textunderscore}BASE/lib/python2.6/site-packages/} to let python find the yarp module.

If you are not using robotpkg to install yarp, then make sure to copy the files \dokumonospace{yarp.py} and \dokumonospace{{\textunderscore}yarp.so} to your Python lib directory (\dokumonospace{/usr/lib/python2.6/site-packages/}) or at some place reachable from your \dokumonospace{PYTHONPATH} environment variable.


\dokutitleleveltree{Pocolibs}
\label{15f13a3fccdd1ef095539316b61c03c8}%% pocolibs

To build Pocolibs bindings (the LAAS-CNRS middleware), you need to install Pocolibs on your system.

The recommended way to do it is through \dokumonospace{robotpkg} (see \href{http://homepages.laas.fr/mallet/robotpkg}{ robotpkg homepage} for more informations).

To install:


\small
\begin{verbatimtab}
$ cd $ROBOTPKG_BASE/robotpkg/devel/pocolibs
$ make update
\end{verbatimtab}
\normalsize

\dokutitleleveltwo{Installation}
\label{ea09bb364ef1bffd889e76b7a59035fc}%% installation

From your MORSE root directory:


\small
\begin{verbatimtab}
$ mkdir build && cd build
$ cmake ..
\end{verbatimtab}
\normalsize

By default, MORSE will install in \dokumonospace{/usr/local}. You can easily change that by launching ccmake instead of cmake.
When using ccmake, it is also possible to select the optional middleware bindings for YARP and Pocolibs.
You can set up the different variables using the command line:


\begin{itemize}
\dokuitem  \dokumonospace{CMAKE{\textunderscore}INSTALL{\textunderscore}PREFIX} controls where will be installed MORSE. Note: The install prefix directory will be refered to as \dokumonospace{{\textdollar}MORSE{\textunderscore}ROOT} in this document.
\dokuitem  \dokumonospace{BUILD{\textunderscore}POCOLIBS{\textunderscore}SUPPORT} controls the build of pocolibs support in MORSE
\dokuitem  \dokumonospace{BUILD{\textunderscore}YARP2{\textunderscore}SUPPORT} controls the build of YARP support in MORSE
\dokuitem  \dokumonospace{CMAKE{\textunderscore}BUILD{\textunderscore}TYPE} controls the optimisation stuff for C/C++ extension (Release is a good choice).
\end{itemize}

\small
\begin{verbatimtab}
$ make install
\end{verbatimtab}
\normalsize

To build and install MORSE with YARP support in \dokumonospace{/opt}, you need something like:


\small
\begin{verbatimtab}
$ cmake -DBUILD_YARP2_SUPPORT=ON -DCMAKE_BUILD_TYPE=Release -DCMAKE_INSTALL_PREFIX=/opt ..
\end{verbatimtab}
\normalsize

Two environment variables must be set to let the simulator know where to look for the installed components:



\begin{itemize}
\dokuitem  \dokumonospace{{\textdollar}PYTHONPATH} specifies the directories where MORSE scripts are located.  It will start with the MORSE installation prefix, followed by the path to the python scripts. It must also include the full path to the directory \dokumonospace{morse/blender} within the installed files. Example:
\end{itemize}

\small
\begin{verbatimtab}
tcsh:
setenv PYTHONPATH
$MORSE_ROOT/lib/python2.6/site-packages:$MORSE_ROOT/lib/python2.6/site-packages/morse/blender
\end{verbatimtab}
\normalsize

\small
\begin{verbatimtab}
bash:
export
PYTHONPATH=$MORSE_ROOT/lib/python2.6/site-packages:$MORSE_ROOT/lib/python2.6/site-packages/morse/blender
\end{verbatimtab}
\normalsize

\begin{itemize}
\dokuitem  {\textdollar}ORS{\textunderscore}BLENDER that allows you to specify a custom Blender executable (else, Blender is looked for in the path).
\end{itemize}

\dokutitleleveltwo{Running a simulation}
\label{70cc733d625ccafb4d56ea8f7dcb7973}%% running_a_simulation

[YARP specific] Before starting a simulation: Start the YARP's server using this command in a separate terminal:


\small
\begin{verbatimtab}
$ yarp server
\end{verbatimtab}
\normalsize

Launch MORSE by calling the morse executable:


\small
\begin{verbatimtab}
$ morse
\end{verbatimtab}
\normalsize

Several options are available, check them with:


\small
\begin{verbatimtab}
$ morse help
\end{verbatimtab}
\normalsize

You can check your configuration is ok with:


\small
\begin{verbatimtab}
$ morse check
\end{verbatimtab}
\normalsize

Once launched, you can test the simulator by loading one of the example scenarii from \dokumonospace{{\textdollar}MORSE{\textunderscore}ROOT/share/examples/morse/scenarii} (.blend files)

To start a simulation, go on Blender and press \key{P} to play the scenario.


\dokutitleleveltwo{Testing}
\label{ae2b1fca515949e5d54fb22b8ed95575}%% testing

To test the external control clients:


\begin{itemize}
\dokuitem  Open the Blender file: \dokumonospace{{\textdollar}MORSE{\textunderscore}ROOT/share/examples/morse/scenarii/example-1-solved.blend}
\dokuitem  Start the simulation \key{P}
\dokuitem  On a separate terminal, run the Python program:
\end{itemize}
  {\textdollar} python {\textdollar}MORSE{\textunderscore}ROOT/share/examples/morse/clients/atrv/socket{\textunderscore}v{\textunderscore}omega{\textunderscore}client.py


\begin{itemize}
\dokuitem  Follow the client program's intructions to send movement commands to the robot and to read information back
\dokuitem  To finish the simulation, press \key{Esc}
\dokuitem  To close Blender, press \key{Ctrl + Q}, and then \key{Enter}
\end{itemize}

Tips: If you have any problem to start to play a simulation: start blender
from a terminal and send the error messages to morse{\textunderscore}dev@laas.fr.

{\dokuquoting}{\dokuquoting} \hyperref[a80da1282f2c775bbc5f2c92c836968b]{ User documentation}


\dokutitlelevelone{Getting started: basic MORSE usage}
\label{cef0f135a902b28b4a51575ed8a2cec1}%% basic_morse_usage
\label{60efe788544a384827c39a9803dab85b}%%Start: basic_morse => /home/slemaign/openrobots/data/pages/basic_morse.txt

\dokutitleleveltwo{Basic Blender commands}
\label{7809ce8524191dac30521dd680b9a379}%% basic_blender_commands

\dokutitleleveltree{Selection of an object}
\label{d461f4bc8beb6b0a8442d6d4da3a0c6d}%% selection_of_an_object
\key{<img src="http://homepages.laas.fr/slemaign/openrobots//lib/plugins/keyboard/images/RMB.png" class="middle" alt="Right mouse button" />} over the object


\dokutitleleveltree{Geometric transformations of an object}
\label{001ad77ecf37a2b719f6d1a347faed00}%% geometric_transformations_of_an_object

\dokutitlelevelfour{Keyborad input}

To enter with the keyboard accurate transformation value, press \key{N}.


\dokutitlelevelfour{With the mouse}
The shortcut keys are \key{G} for translation, \key{S} for scaling and
 \key{R} for rotation.

Press the desired key, then move with the mouse.

Press \includegraphics[height=1em]{LMB.png} to accept transformation, or \includegraphics[height=1em]{RMB.png} to cancel.

Tips:


\begin{itemize}
\dokuitem  Hold \key{Ctrl} key while transforming, to use discrete intervals. 
\dokuitem  Press \key{X}, \key{Y} or \key{Z} to constraint the transformation on one axis.
\end{itemize}

\dokutitleleveltree{Camera view shortcuts}
\label{490b5cdd607bc3b8891df904de25d417}%% camera_view_shortcuts
Use the numeric keyboard


\begin{itemize}
\dokuitem  Top view: \key{'7 NumPad'}
\dokuitem  Front view: \key{'1 NumPad'}
\dokuitem  Side view: \key{'3 NumPad'}
\dokuitem  Main camera view: \key{'0 NumPad'}
\dokuitem  Toggle perspective/orthogonal view: \key{'5 NumPad'}
\end{itemize}

\dokutitleleveltree{Switch shading modes}
\label{930ca388b865b6b10e3f5eb29f9bb084}%% switch_shading_modes

\begin{itemize}
\dokuitem  Wireframe/solid: \key{Z}
\dokuitem  Wireframe/solid: \key{Shift + Z}
\dokuitem  Solid/textured: \key{Alt + Z}
\end{itemize}

\dokutitleleveltree{Opening and saving files}
\label{0b2dd5c17c378b3048db1b45587a2d17}%% opening_and_saving_files

\begin{itemize}
\dokuitem  Open file: \key{F1}
\dokuitem  Save file: \key{Ctrl + W}
\dokuitem  Save file as\ldots{} : \key{F2}
\dokuitem  Quit Blender: \key{Ctrl + Q}
\end{itemize}

\dokutitleleveltree{Button pannels}
\label{ec1467299d35fb7adfeb7b88dce3d414}%% button_pannels
Selection of the type of pannel in the Buttons Window:


\begin{itemize}
\dokuitem  Logic buttons: \key{F4}
\dokuitem  Material buttons: \key{F5}
\dokuitem  Texture buttons: \key{F6}
\dokuitem  Object buttons: \key{F7}
\dokuitem  Editting buttons: \key{F9}
\end{itemize}

\dokutitleleveltwo{Creation of scenario files}
\label{477e85464263e092bdd386389bab76b2}%% creation_of_scenario_files

Use the \dokumonospace{morse} binary program, to open or create new simulation scenarios:



\small
\begin{verbatimtab}
$ morse
\end{verbatimtab}
\normalsize

Will create a new copy of the basic scenario file, ready to be modified.



\small
\begin{verbatimtab}
$ morse create [Filename]
\end{verbatimtab}
\normalsize

It will create a new file with the name given, and ready to be edited.



\small
\begin{verbatimtab}
$ morse [Filename]
\end{verbatimtab}
\normalsize

The MORSE simulator is started using an existing file with the name specified.


\dokutitleleveltwo{Adding components to files}
\label{94afd3e6d92cbe0e7c018d7a85463367}%% adding_components_to_files

\begin{itemize}
\dokuitem  Open file
\dokuitem  Link elements (robots, scenarios sensors) \key{Shift + F1}
\dokuitem  Select the source Blender file, and then Objects
\dokuitem  Select the objects to insert, using the \key{<img src="http://homepages.laas.fr/slemaign/openrobots//lib/plugins/keyboard/images/RMB.png" class="middle" alt="Right mouse button" />}
\dokuitem  Select the \dokubold{Link} and \dokubold{Relative Paths} toggle buttons
\dokuitem  Click \dokubold{Load Library}
\dokuitem  The objects will be inserted in the scene, at the origin, and with a cyan selection highlight
\dokuitem  Select the parent object, then press \key{Shift + G}, then select \dokubold{Children} or press \key{Enter}
\dokuitem  Press the \key{L} key, then select \dokubold{Selected Objects} or press \key{Enter}
\dokuitem  The object will now have a pink selection highlight, and can be moved/rotated/scaled
\end{itemize}

\dokutitleleveltwo{Parenting components to a robot}
\label{5c8528e5d80c7c0537856e0911e6cdc8}%% parenting_components_to_a_robot

\begin{itemize}
\dokuitem  Select the Empty object of the component, then hold \key{Shift} and select the robot
\dokuitem  Press \key{Ctrl + P}, then select \dokubold{Make parent} or press \key{Enter}
\end{itemize}

\dokutitleleveltwo{Start and stop a simulation}
\label{466b971dc2d338787d665353c2a0f40a}%% start_and_stop_a_simulation

To start the simulation, press the \key{P} key.

To stop, press the \key{Esc} key.

\includegraphics[height=1em]{attention.png} Under certain error conditions ESC may not work, in that case, an emergency exit
key is \key{F12}, wich will try to exit without doing executing the 
cleanup functions. If this doesn't work, the command \key{Ctrl + \\} on the 
terminal will kill Blender.

{\dokuquoting}{\dokuquoting} \hyperref[a80da1282f2c775bbc5f2c92c836968b]{ User documentation}


\dokutitlelevelone{Tutorial example}
\label{d0e1bed8b40ec8e4f9cc0cc0a11ff110}%% tutorial_example
\label{0575c8d592fb7b088226750aceec2b4e}%%Start: tutorial => /home/slemaign/openrobots/data/pages/tutorial.txt

This tutorial assumes MORSE is properly installed. If not, follow the instructions \hyperref[ea09bb364ef1bffd889e76b7a59035fc]{ here}.


\dokutitleleveltwo{Setup of the simulation scene}
\label{80527725485ea9e7bedbc9d918895a02}%% setup_of_the_simulation_scene

\dokutitleleveltree{Load sample file}
\label{4238ab1e3d8f80f2fdec7f34e17e0f67}%% load_sample_file

Open the MORSE simulator with the test file provided with the installation, by using this command:



\small
\begin{verbatimtab}
$ morse $ORS_ROOT/share/examples/morse/scenarii/example-1.blend
\end{verbatimtab}
\normalsize

This will load a scene with a robot in a room with some furniture.


\dokutitleleveltree{Link an actuator}
\label{81c6b973417d3e5000d27d9c1c805b96}%% link_an_actuator

We'll add a motion controller to the robot, so that it can receive commands from an external program. The robot will then move according to the instructions received. In this case we'll add a controller that uses linear and angular speed (V, W).



\begin{enumerate}\dokuitem  With the mouse over the 3D view in Blender, press \key{Shift + F1} to open the Load Library browser
\dokuitem  Navigate to the directory \dokumonospace{{\textdollar}ORS{\textunderscore}ROOT/data/morse/components/controllers}
\dokuitem  Press \key{<img src="http://homepages.laas.fr/slemaign/openrobots//lib/plugins/keyboard/images/LMB.png" class="middle" alt="Left mouse button" />} over the file \dokumonospace{morse{\textunderscore}vw{\textunderscore}control.blend}
\dokuitem  Press \key{<img src="http://homepages.laas.fr/slemaign/openrobots//lib/plugins/keyboard/images/LMB.png" class="middle" alt="Left mouse button" />} over the item \dokumonospace{Object}
\dokuitem  Toggle the buttons \dokubold{Relative Paths} and \dokubold{Link} at the bottom of the window
\dokuitem  Press \key{<img src="http://homepages.laas.fr/slemaign/openrobots//lib/plugins/keyboard/images/RMB.png" class="middle" alt="Right mouse button" />} over the item \dokumonospace{Motion{\textunderscore}Controller}
\dokuitem  Press the button \dokubold{Load Library}. You'll return to the 3D View
\dokuitem  Select the newly inserted object in the scene, either by \key{<img src="http://homepages.laas.fr/slemaign/openrobots//lib/plugins/keyboard/images/RMB.png" class="middle" alt="Right mouse button" />} clicking over the object in the 3D View, or \key{<img src="http://homepages.laas.fr/slemaign/openrobots//lib/plugins/keyboard/images/LMB.png" class="middle" alt="Left mouse button" />} over the object's name in the Outliner window. The object will be highlighted in cyan colour, and can not be moved around.
\dokuitem  Convert the object to local, by pressing \key{L} then hitting \key{Enter}
\dokuitem  With the controller selected, hold down \key{Shift} and then \key{<img src="http://homepages.laas.fr/slemaign/openrobots//lib/plugins/keyboard/images/RMB.png" class="middle" alt="Right mouse button" />} over the robot object
\dokuitem  Press \key{Ctrl + P} and then hit \key{Enter} make the robot the parent of the controller
\end{enumerate}

\dokutitleleveltree{Link a Gyroscope sensor}
\label{f019fe80659ff060e77872347a3add5c}%% link_a_gyroscope_sensor

Next we'll add a sensor to the robot that will report the angles of the robot orientation with respect to the reference axes (yaw, pitch and roll)



\begin{enumerate}\dokuitem  With the mouse over the 3D view in Blender, press \key{Shift + F1} to open the Load Library browser
\dokuitem  Navigate to the directory \dokumonospace{{\textdollar}ORS{\textunderscore}ROOT/data/morse/components/sensors}
\dokuitem  Press \key{<img src="http://homepages.laas.fr/slemaign/openrobots//lib/plugins/keyboard/images/LMB.png" class="middle" alt="Left mouse button" />} over the file \dokumonospace{morse{\textunderscore}gyroscope.blend}
\dokuitem  Press \key{<img src="http://homepages.laas.fr/slemaign/openrobots//lib/plugins/keyboard/images/LMB.png" class="middle" alt="Left mouse button" />} over the item \dokumonospace{Object}
\dokuitem  Toggle the buttons \dokubold{Relative Paths} and \dokubold{Link} at the bottom of the window
\dokuitem  Press \key{<img src="http://homepages.laas.fr/slemaign/openrobots//lib/plugins/keyboard/images/RMB.png" class="middle" alt="Right mouse button" />} over the items \dokumonospace{Gyroscope} and \dokumonospace{Gyro{\textunderscore}box}
\dokuitem  Press the button \dokubold{Load Library}. You'll return to the 3D View
\dokuitem  Select the newly inserted \dokumonospace{Gyroscope} object in the scene, either by \key{<img src="http://homepages.laas.fr/slemaign/openrobots//lib/plugins/keyboard/images/RMB.png" class="middle" alt="Right mouse button" />} clicking over the object in the 3D View, or \key{<img src="http://homepages.laas.fr/slemaign/openrobots//lib/plugins/keyboard/images/LMB.png" class="middle" alt="Left mouse button" />} over the object's name in the Outliner window. The object will be highlighted in cyan colour, and can not be moved around.
\dokuitem  Select the child object, by pressing \key{Shift + G}, then hitting \key{Enter}
\dokuitem  Convert the object to local, by pressing \key{L} then hitting \key{Enter}
\dokuitem  Switch to front view by pressing \key{'1 NumPad'}
\dokuitem  Press \key{G}, then move the \dokumonospace{Gyroscope} object to the correct location with respect to the robot
\dokuitem  Press \key{<img src="http://homepages.laas.fr/slemaign/openrobots//lib/plugins/keyboard/images/LMB.png" class="middle" alt="Left mouse button" />} to accept the movement
\dokuitem  With the \dokumonospace{Gyroscope} object selected, hold down \key{Shift} and then \key{<img src="http://homepages.laas.fr/slemaign/openrobots//lib/plugins/keyboard/images/RMB.png" class="middle" alt="Right mouse button" />} over the robot object
\dokuitem  Press \key{Ctrl + P} and then hit \key{Enter} make the robot the parent of the controller
\end{enumerate}

\dokutitleleveltree{Insert the middleware object}
\label{fc2213e90f6f9853c66c14f9f79c3379}%% insert_the_middleware_object
To use a middleware to exchange data from the simulator, it is necessary to link in an object that will represent the middleware.



\begin{enumerate}\dokuitem  With the mouse over the 3D view in Blender, press \key{Shift + F1} to open the Load Library browser
\dokuitem  Navigate to the directory \dokumonospace{{\textdollar}ORS{\textunderscore}ROOT/data/morse/components/middleware}
\dokuitem  Press \key{<img src="http://homepages.laas.fr/slemaign/openrobots//lib/plugins/keyboard/images/LMB.png" class="middle" alt="Left mouse button" />} over the file \dokumonospace{socket{\textunderscore}empty.blend}
\dokuitem  Press \key{<img src="http://homepages.laas.fr/slemaign/openrobots//lib/plugins/keyboard/images/LMB.png" class="middle" alt="Left mouse button" />} over the item \dokumonospace{Object}
\dokuitem  Toggle the buttons \dokubold{Relative Paths} and \dokubold{Link} at the bottom of the window
\dokuitem  Press \key{<img src="http://homepages.laas.fr/slemaign/openrobots//lib/plugins/keyboard/images/RMB.png" class="middle" alt="Right mouse button" />} over the item \dokumonospace{Socket{\textunderscore}Empty}
\dokuitem  Press the button \dokubold{Load Library}. You'll return to the 3D View
\dokuitem  It is not necessary to make this object local or to move it. But it can be useful to avoid cluttering of items in the scene
\end{enumerate}

\dokuunderline{Note}: One single middleware Empty is necessary to enable the middleware, regardless of how many components will make use of it.


\dokutitleleveltree{Configuring the middlewares}
\label{7c1b9786b6402b908e3042548cd3c1c6}%% configuring_the_middlewares
Binding the components in the scene with the middleware is done in a configuration file within the Blender file.



\begin{enumerate}\dokuitem  On the \dokubold{Text Editor} window, select the file \dokumonospace{component{\textunderscore}config.py}
\dokuitem  Add the following items to the \dokumonospace{component{\textunderscore}mw} dictionary:
\end{enumerate}

\lstset{language=python}
\begin{lstlisting}
component_mw = {
    "Gyroscope": ["Socket", "post_message"],
    "Motion_Controller": ["Socket", "read_message"],
	}

\end{lstlisting}

\dokutitleleveltree{Run the simulation}
\label{62874528899bc63c891e142b192d89b7}%% run_the_simulation
Press \key{P} to start the Game Engine


\dokutitleleveltree{Connect with the client}
\label{646d760bcdaf8445e5fb3dbad2c443e5}%% connect_with_the_client
Use the example client program to test the bindings in the simulation



\begin{enumerate}\dokuitem  On a separate terminal, navigate to the directory \dokumonospace{{\textdollar}ORS{\textunderscore}ROOT/share/examples/morse/clients/atrv/}
\dokuitem  Execute the command
\end{enumerate}
    {\textdollar} python socket{\textunderscore}v{\textunderscore}omega{\textunderscore}client.py


\begin{enumerate}\dokuitem  Press \key{A} to give speed commands to the robot
\dokuitem  Type a the linear and angular speeds, followed by \key{Enter} after each
\dokuitem  The robot should start moving in MORSE
\dokuitem  Press \key{B} to print the readings of the gyroscope exported by MORSE
\dokuitem  Press \key{Q} to exit the client
\end{enumerate}

Finally exit the simulation, by pressing \key{Esc} on the Blender window, then close Blender by pressing \key{Ctrl + Q}, then \key{Enter}.

{\dokuquoting}{\dokuquoting} \hyperref[a80da1282f2c775bbc5f2c92c836968b]{ User documentation}


\dokutitlelevelone{Building an equiped robot}
\label{23e1989e63bbd01247feb36cca0a69e8}%% building_an_equiped_robot
\label{feb94730bf2c8bc6803a472bb56691ae}%%Start: equip_robot => /home/slemaign/openrobots/data/pages/equip_robot.txt

This tutorial provides instructions on how to create a file for a robot prepared with a defined number of components, which can later be inserted into scenario files.


\dokutitleleveltwo{Setup of the robot file}
\label{bc8d7eea217f972d09e6a6c0b666fd98}%% setup_of_the_robot_file

Create a new MORSE file. Give it a name that represents the settings of the robot. As an example this document will use the Ressac helicopter:



\small
\begin{verbatimtab}
$ morse create $ORS_ROOT/share/data/morse/components/robots/ressac_equiped.blend
\end{verbatimtab}
\normalsize

Next link in the base of the robot:



\begin{enumerate}\dokuitem  With the mouse over the 3D view in Blender, press \key{Shift + F1} to open the Load Library browser
\dokuitem  Navigate to the directory \dokumonospace{{\textdollar}ORS{\textunderscore}ROOT/data/morse/components/robots}
\dokuitem  Press \key{<img src="http://homepages.laas.fr/slemaign/openrobots//lib/plugins/keyboard/images/LMB.png" class="middle" alt="Left mouse button" />} over the file \dokumonospace{ressac.blend}
\dokuitem  Press \key{<img src="http://homepages.laas.fr/slemaign/openrobots//lib/plugins/keyboard/images/LMB.png" class="middle" alt="Left mouse button" />} over the item \dokumonospace{Object}
\dokuitem  Toggle the buttons \dokubold{Relative Paths} and \dokubold{Link} at the bottom of the window
\dokuitem  Press \key{<img src="http://homepages.laas.fr/slemaign/openrobots//lib/plugins/keyboard/images/RMB.png" class="middle" alt="Right mouse button" />} and drag over the names of all the objects listed, to select them all
\dokuitem  Press the button \dokubold{Load Library}. You'll return to the 3D View
\dokuitem  Select the newly inserted objects in the scene, either by \key{<img src="http://homepages.laas.fr/slemaign/openrobots//lib/plugins/keyboard/images/RMB.png" class="middle" alt="Right mouse button" />} clicking over the \dokumonospace{Ressac} object in the 3D View, or \key{<img src="http://homepages.laas.fr/slemaign/openrobots//lib/plugins/keyboard/images/LMB.png" class="middle" alt="Left mouse button" />} over the object's name in the Outliner window. The object will be highlighted in cyan colour, and can not be moved around.
\dokuitem  Select the child objects, by pressing \key{Shift + G}, then hitting \key{Enter}
\dokuitem  Convert the objects to local, by pressing \key{L} then hitting \key{Enter}
\end{enumerate}

The rest of the components (sensors and actuators) should be linked similarly. Refer to the \hyperref[0575c8d592fb7b088226750aceec2b4e]{ Quick tutorial} for instructions. In the case of a robot file, no middlewares or modifiers should be added, since those would be specific to every particular scenario.

Adjust the properties of the component if necessary. Then save the file again, by pressing \key{Ctrl + W}, followed by \key{Enter}.

This robot file should be liked into scenarii files by following the same procedure, while selecting all the objects contained in the file.

{\dokuquoting}{\dokuquoting} \hyperref[a80da1282f2c775bbc5f2c92c836968b]{ User documentation}


\dokutitlelevelone{The MORSE user workflow}
\label{2eb6dd59bcc0f89edd80bfe2fc4bec7f}%% the_morse_user_workflow
\label{514bac84019bd5e09c0e2b525b09f429}%%Start: user_workflow => /home/slemaign/openrobots/data/pages/user_workflow.txt

The general steps to follow to do a simulation in MORSE are the following:



\begin{itemize}
\dokuitem  Create the robot configuration that matches the real robot
\dokuitem  Create a scenario file, based on the default MORSE file
\end{itemize}
    {\textdollar} morse create [file{\textunderscore}name]


\begin{itemize}
\dokuitem  Link in the robot
\dokuitem  Link in any aditional robots/sensors/actuators
\dokuitem  Link in the middleware and modifier objects
\dokuitem  Configure the components to use specific modifiers and middlewares
\dokuitem  Save the new scene
\dokuitem  Prepare the middleware environments (in the case of YARP or Pocolibs)
\dokuitem  Start the simulation
\dokuitem  Look at the terminal window to check that everything started correctly
\dokuitem  Start the client programs that will connect with the simulator
\dokuitem  Finish the simulation by pressing \key{Esc}
\dokuitem  Quit Blender by pressing \key{Ctrl + Q} and \key{Enter}
\end{itemize}
{\dokuquoting}{\dokuquoting} \hyperref[a80da1282f2c775bbc5f2c92c836968b]{ User documentation}


\dokutitlelevelone{The MORSE Component Library}
\label{1d548ae3d571b50e147fca31be425d0f}%% the_morse_component_library
\label{004fdec0cc1a00c19c57e892b7eb1400}%%Start: component_library => /home/slemaign/openrobots/data/pages/component_library.txt

The modular concept of morse is based on individual components with specific functions that replicate the parts of a real robot. Components in MORSE can belong to three main types:


\begin{itemize}
\dokuitem  \hyperref[d69ac14cd721dd995822d4e984f48116]{ Robots}
\dokuitem  \hyperref[029aee483db9ae244d7a5cb353e74602]{ Sensors}
\dokuitem  \hyperref[2068e59180763f350d66a42e828e7f96]{ Actuators}
\end{itemize}

In general, Sensors generate data from the simulated world, to be used by external programs. Alternatively, Actuators read data from outside Blender to be applied inside the simulation. Both Sensors and Actuators must be linked to a Robot to function.
Each component has its own data structure with the variables it requires to function.

The data generated inside Blender can be considered to be "perfect", because it is very accurate and is not affected by real phenomenons. To make the data generated more realistic, it is possible to use \hyperref[bf24b44a8cc99e648657b164c8aba758]{ Modifiers} that will alter the data collected by the sensors. The modifiers can add noise, apply corrections or delays, or change the data with respect to other criteria.

To interact with the outside world, components rely on Middlewares connected to Blender. See also \hyperref[9a05db9c4b60b0527010fd997682f523]{ the list of supported middlewares} for a further explanation.


\dokutitleleveltwo{Sensors}
\label{029aee483db9ae244d7a5cb353e74602}%% sensors

\begin{itemize}
\dokuitem  \hyperref[11648e4e66e7ed6a86cb7f1d0cf604fe]{ GPS}
\dokuitem  \hyperref[6b3b2d8500522343e080755f0e0aa4fe]{ Gyroscope}
\dokuitem  \hyperref[dd6d2dcc679d12b9430a9787bab45b33]{ Video camera}
\dokuitem  \hyperref[8d7d5ffd0031f2449cbeaef424c22d75]{ SICK laser}
\end{itemize}

\dokutitleleveltwo{Actuators/Controllers}
\label{2068e59180763f350d66a42e828e7f96}%% controllers

\begin{itemize}
\dokuitem  \hyperref[388a56dbb62a010dc26a378981346247]{ Keyboard arrows}
\dokuitem  \hyperref[cdf7afd8bc8dbb764b14c987cea8effd]{ Linear and angular speed (V, W)}
\dokuitem  \hyperref[6990a54322d9232390a784c5c9247dd6]{ Straight line movement}
\dokuitem  \hyperref[f75862c2bd0040eb683048c313dcaaa8]{ Waypoint destination}
\end{itemize}

\dokutitleleveltwo{Bare robotic bases}
\label{d69ac14cd721dd995822d4e984f48116}%% bare_robotic_bases

\begin{itemize}
\dokuitem  \hyperref[4fd87f5742582d412dce2c6ad5304937]{ iRobot ATRV}
\dokuitem  \hyperref[311954cf2f831f2289fb7fff75d15a7d]{ Yamaha RMax}
\dokuitem  \hyperref[3c16132d99703978dacd02b0808a4270]{ NeoBotix platform with PA-10 robotic arm}
\end{itemize}
{\dokuquoting}{\dokuquoting} \hyperref[a80da1282f2c775bbc5f2c92c836968b]{ User documentation}


\dokutitleleveltwo{Modifiers}
\label{bf24b44a8cc99e648657b164c8aba758}%% modifiers
\label{25bc6523e9298f4691b3c8200a395d92}%%Start: modifier_introduction => /home/slemaign/openrobots/data/pages/modifier_introduction.txt

Modifiers affect directly the data employed by sensors and actuators, and are specific to the data used by the components. Just like middlewares, they must implement a method called \dokumonospace{register{\textunderscore}component} that should add the corresponding function to the component's action list.


\dokutitleleveltwo{List of existing modifiers}
\label{e1bd7dc12fc91796f6afa908960bddfd}%% list_of_existing_modifiers

\begin{itemize}
\dokuitem  \hyperref[b32d6491ce03dd4e6c877f3bfd9ff07e]{ UTM conversion}
\dokuitem  \hyperref[f68daad189b2fffd0b8cab5e36ec9d96]{ NED conversion}
\dokuitem  \hyperref[466deec76ecdf5fca6d38571f6324d54]{ JSON encoding/decoding}
\dokuitem  \hyperref[aa061f2a51a69b41bf030cab71d64ed9]{ GPS noise}
\end{itemize}

\dokutitleleveltwo{Linking a modifier in a scene}
\label{840a8bd36b58f7e792398436a4be45db}%% linking_a_modifier_in_a_scene

To be able to use a modifier inside of a scene, it is necessary to link the Empty object from the corresponding Blender file. This process is identical to the one used for middlewares, as explained in the \hyperref[0575c8d592fb7b088226750aceec2b4e]{ basic tutorial}, together with an explanation on how to configure the components to call middleware functions.


\dokutitleleveltwo{Creating a new modifier}
\label{2f5deef38336990fa7f864f2028d68a9}%% creating_a_new_modifier

The concept of a modifier is relatively simple. Their only function is to change the data stored in variables in the corresponding component, by using the concept of \hyperref[4e819c837d54a6ed09abc77a8560a66f]{hooks}. Creating a new modifier consists mainly of writing the Python script that will alter the data. The modifier should only work on the \dokumonospace{modified{\textunderscore}data} array of a MORSE component, and it is important to ensure the fields of the array are maintained with the same order and data type.

{\dokuquoting}{\dokuquoting} \hyperref[a80da1282f2c775bbc5f2c92c836968b]{ User documentation}


\dokutitlelevelone{Middleware Support}
\label{4303941a1597ae94654bd96854480742}%% middleware_support
\label{9a05db9c4b60b0527010fd997682f523}%%Start: supported_middlewares => /home/slemaign/openrobots/data/pages/supported_middlewares.txt

Middlewares provide a means for the simulated data to be shared with external programs. MORSE is designed to be middleware independent, so that its internal functioning is not tied no any one particular middleware, but it is capable of communicating with any type of architecture.


\dokutitleleveltwo{Current list of compatible middlewares}
\label{92515de7e8c9f43d6ca122cbbfd1809e}%% current_list_of_compatible_middlewares

\begin{itemize}
\dokuitem  \hyperref[1cb251ec0d568de6a929b520c4aed8d1]{ Text file output}
\dokuitem  \hyperref[61f2529360aec54f5dc9804b842cf3fa]{ Sockets}
\dokuitem  \hyperref[ec46d0b85077d7a7fe8da2e2b4c70462]{ YARP}
\dokuitem  \hyperref[15f13a3fccdd1ef095539316b61c03c8]{ Pocolibs}
\end{itemize}

\dokutitleleveltwo{Linking a middleware in a scene}
\label{5ded332fc3ba470e4d4d290c9bf26a19}%% linking_a_middleware_in_a_scene

To be able to use a middleware inside of a scene, it is necessary to link the Empty object from the corresponding Blender file. This process is explained in the \hyperref[0575c8d592fb7b088226750aceec2b4e]{ basic tutorial}. Also there is an explanation on how to configure the components to call middleware functions.


\dokutitleleveltwo{Expanding the middlewares}
\label{b3a6313d335453f4c7ad970485acc1a1}%% expanding_the_middlewares

New middlewares can be added to MORSE by following these \hyperref[6a8f80abb2f3d2288ad863e67f2499a4]{ instructions}.

{\dokuquoting}{\dokuquoting} \hyperref[a80da1282f2c775bbc5f2c92c836968b]{ User documentation}


\dokutitlelevelone{Advanced tutorials}
\label{1db3103f04a8f50e1168ef3c23748f71}%% advanced_tutorials
\label{1db3103f04a8f50e1168ef3c23748f71}%%Start: advanced_tutorials => /home/slemaign/openrobots/data/pages/advanced_tutorials.txt

\begin{itemize}
\dokuitem  \hyperref[feb94730bf2c8bc6803a472bb56691ae]{ Preparing a robot with specific equipment}
\dokuitem  \hyperref[1dd029a60f7f3dd1deaf993ce4538edf]{ YARP-based simulation tutorial}
\dokuitem  \hyperref[5c7d3aeca93d2be4626b023df992dc1d]{ Pocolibs (Genom) tutorial}
\end{itemize}
{\dokuquoting}{\dokuquoting} \hyperref[a80da1282f2c775bbc5f2c92c836968b]{ User documentation}  \hyperref[1db3103f04a8f50e1168ef3c23748f71]{ Advanced tutorials}


\dokutitlelevelone{YARP-based simulation tutorial}
\label{46a8ae159056a35cad5aad3f96f08029}%% yarp-based_simulation_tutorial
\label{1dd029a60f7f3dd1deaf993ce4538edf}%%Start: yarp_tutorial => /home/slemaign/openrobots/data/pages/yarp_tutorial.txt

\dokutitleleveltwo{Setup}
\label{a0f848942ce863cf53c0fa6cc684007d}%% setup

You need to install YARP and its Python bindings, by following the \hyperref[ec46d0b85077d7a7fe8da2e2b4c70462]{ instructions} in the installation page.

Before running a simulation using YARP, it is necessary to open a new shell terminal and start the \dokumonospace{yarpserver} program:


\small
\begin{verbatimtab}
$ yarpserver
\end{verbatimtab}
\normalsize

\dokutitleleveltwo{Configuring the scenario}
\label{a5eb0127854ee2548f6841c01cbaee73}%% configuring_the_scenario

You must link a YARP middleware object into the MORSE scenario file
Create the bidings of the components with yarp, by editing the file \dokumonospace{component{\textunderscore}config.py} inside the Blender file.


\dokutitleleveltree{Link a Camera sensor}
\label{2e2e4de5af03ad71dd248a06e314e9d7}%% link_a_camera_sensor

\begin{enumerate}\dokuitem  With the mouse over the 3D view in Blender, press \key{Shift + F1} to open the Load Library browser
\dokuitem  Navigate to the directory \dokumonospace{{\textdollar}ORS{\textunderscore}ROOT/data/morse/components/sensors}
\dokuitem  Press \key{<img src="http://homepages.laas.fr/slemaign/openrobots//lib/plugins/keyboard/images/LMB.png" class="middle" alt="Left mouse button" />} over the file \dokumonospace{morse{\textunderscore}camera.blend}
\dokuitem  Press \key{<img src="http://homepages.laas.fr/slemaign/openrobots//lib/plugins/keyboard/images/LMB.png" class="middle" alt="Left mouse button" />} over the item \dokumonospace{Object}
\dokuitem  Toggle the buttons \dokubold{Relative Paths} and \dokubold{Link} at the bottom of the window
\dokuitem  Press \key{<img src="http://homepages.laas.fr/slemaign/openrobots//lib/plugins/keyboard/images/RMB.png" class="middle" alt="Right mouse button" />} over the items \dokumonospace{CameraMain}, \dokumonospace{CameraUser}, \dokumonospace{CameraCube}, \dokumonospace{CameraLens}
\dokuitem  Press the button \dokubold{Load Library}. You'll return to the 3D View
\dokuitem  Select the newly inserted \dokumonospace{CameraMain} object in the scene, either by \key{<img src="http://homepages.laas.fr/slemaign/openrobots//lib/plugins/keyboard/images/RMB.png" class="middle" alt="Right mouse button" />} clicking over the object in the 3D View, or \key{<img src="http://homepages.laas.fr/slemaign/openrobots//lib/plugins/keyboard/images/LMB.png" class="middle" alt="Left mouse button" />} over the object's name in the Outliner window. The object will be highlighted in cyan colour, and can not be moved around.
\dokuitem  Select the child object, by pressing \key{Shift + G}, then hitting \key{Enter}
\dokuitem  Convert the object to local, by pressing \key{L} then hitting \key{Enter}
\dokuitem  Switch to front view by pressing \key{'1 NumPad'}
\dokuitem  Press \key{G}, then move the \dokumonospace{CameraMain} object to the correct location with respect to the robot
\dokuitem  Press \key{<img src="http://homepages.laas.fr/slemaign/openrobots//lib/plugins/keyboard/images/LMB.png" class="middle" alt="Left mouse button" />} to accept the movement
\dokuitem  With the \dokumonospace{CameraMain} object selected, hold down \key{Shift} and then \key{<img src="http://homepages.laas.fr/slemaign/openrobots//lib/plugins/keyboard/images/RMB.png" class="middle" alt="Right mouse button" />} over the robot object
\dokuitem  Press \key{Ctrl + P} and then hit \key{Enter} make the robot the parent of the controller
\end{enumerate}

\dokutitleleveltree{Insert the middleware object}
\label{fc2213e90f6f9853c66c14f9f79c3379}%% insert_the_middleware_object

\begin{enumerate}\dokuitem  With the mouse over the 3D view in Blender, press \key{Shift + F1} to open the Load Library browser
\dokuitem  Navigate to the directory \dokumonospace{{\textdollar}ORS{\textunderscore}ROOT/data/morse/components/middleware}
\dokuitem  Press \key{<img src="http://homepages.laas.fr/slemaign/openrobots//lib/plugins/keyboard/images/LMB.png" class="middle" alt="Left mouse button" />} over the file \dokumonospace{yarp{\textunderscore}empty.blend}
\dokuitem  Press \key{<img src="http://homepages.laas.fr/slemaign/openrobots//lib/plugins/keyboard/images/LMB.png" class="middle" alt="Left mouse button" />} over the item \dokumonospace{Object}
\dokuitem  Toggle the buttons \dokubold{Relative Paths} and \dokubold{Link} at the bottom of the window
\dokuitem  Press \key{<img src="http://homepages.laas.fr/slemaign/openrobots//lib/plugins/keyboard/images/RMB.png" class="middle" alt="Right mouse button" />} over the item \dokumonospace{Yarp{\textunderscore}Empty}
\dokuitem  Press the button \dokubold{Load Library}. You'll return to the 3D View
\dokuitem  It is not necessary to make this object local or to move it. But it can be useful to avoid cluttering of items in the scene 
\end{enumerate}

\dokuunderline{Note}: One single middleware Empty is necessary to enable the middleware, regardless of how many components will make use of it.


\dokutitleleveltree{Configuring the middlewares}
\label{7c1b9786b6402b908e3042548cd3c1c6}%% configuring_the_middlewares
Binding the components in the scene with the middleware is done in a configuration file within the Blender file.



\begin{enumerate}\dokuitem  On the \dokubold{Text Editor} window, select the file \dokumonospace{component{\textunderscore}config.py}
\dokuitem  Add the following items to the \dokumonospace{component{\textunderscore}mw} dictionary:
\begin{enumerate}\dokuitem  \dokumonospace{"CameraMain": ["Yarp", "post{\textunderscore}image{\textunderscore}RGBA"],}
\dokuitem  \dokumonospace{"GPS": ["Yarp", "post{\textunderscore}message"],}
\dokuitem  \dokumonospace{"Motion{\textunderscore}Controller": ["Yarp", "read{\textunderscore}message"],}
\end{enumerate}

\end{enumerate}

\dokutitleleveltwo{Reading/writing data}
\label{4531cd1c3fba04d65475a4caadd2beb1}%% writing_data

When the simulation starts, it will print the names of the YARP ports that have been created for every corresponding component. These port names can be used to connect to the component from an external program or client.

The simplest method to test the reading and writing of data is by using the termianl clients. For example, to read the GPS data of the robot through a port named \dokumonospace{/ors/robots/OBATRV/OBGPS/out}, you can type the following in a terminal:


\small
\begin{verbatimtab}
$ yarp read /data/in /ors/robots/OBATRV/OBGPS/out
\end{verbatimtab}
\normalsize

To enter speed commands through a port named \dokumonospace{/ors/robots/OBATRV/OBMotion{\textunderscore}Controller/in}, use the command


\small
\begin{verbatimtab}
$ yarp write /data/out /ors/robots/OBATRV/OBMotion_Controller/in
\end{verbatimtab}
\normalsize
Then type the three destination coordinates, separated by spaces, and press \key{Enter}

To view the images of the camera though a port \dokumonospace{/ors/robots/OBATRV/OBCameraMain/out}:


\small
\begin{verbatimtab}
$ yarpview /img/read &
$ yarp connect /ors/robots/OBATRV/OBCameraMain/out /img/read
\end{verbatimtab}
\normalsize
\end{document}

